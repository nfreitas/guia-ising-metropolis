\documentclass[a4paper,11pt,oneside]{article}
\pagestyle{plain}
%\usepackage[a4paper, margin=1.2in]{geometry}

%configuración del idioma y la codificacion
\usepackage[spanish]{babel}
\usepackage[utf8x]{inputenc}
\usepackage[T1]{fontenc}

%fuente
%\usepackage{times}
\usepackage{palatino}

%math
\usepackage{amsmath,amssymb}
\usepackage{amsthm}
\usepackage{bbold}
\usepackage{dsfont}
\usepackage{braket,mleftright}

\usepackage{graphicx}
\usepackage{subcaption}

%otros paquetes
%\usepackage{rotating}
\usepackage{cite}
\usepackage[linktoc=page]{hyperref}
\usepackage{indentfirst}

%comandos
\newcommand{\mean}[1]{\langle #1 \rangle}

\title{Práctica computacional - Modelo de Ising}
\author{}
\date{}

\begin{document}
\maketitle

\section{Introducción}

Consideremos una grilla de $N = L\times L$ sitios en cada uno de los cuales hay
un spin que solo puede tomar los valores $S = 1,-1$. Estos
spines tienen interacciones tipo Ising dadas por el siguiente Hamiltoniano:
\begin{equation}
    H = -J\sum_\text{n.n} S_i S_j - H \sum_i S_i
\end{equation}
Donde la primer suma es sobre primeros vecinos y $J$ y $H$ son parámetros que
dan la energía de interacción entre spines, y entre cada uno de ellos y un
campo magnético externo. Supongamos que queremos calcular numéricamente algún
observable macroscópico a una dada temperatura $\beta^{-1}$, como por ejemplo
el valor medio de la magnetización total:
\begin{equation}
    \mean{M} = \sum_{S_1,\dots,S_N} p_{S_1,\dots,S_N} \; (S_1+S_2+\dots+S_N).
    \label{eq:mag_total}
\end{equation}
Esta vez la suma es sobre todas las configuraciones posibles de los spines, y
$p_{S_1,\dots,S_N}$ es la probabilidad de cada configuración, que según el
ensamble canónico, es:
\begin{equation}
    p_{S_1,\dots,S_N} = \frac{1}{Z} \; e^{-\beta H(S_1,\dots,S_N)},
\end{equation}
donde $Z$ es la función de partición:
\begin{equation}
    Z = \sum_{S_1,\dots,S_N} e^{-\beta H(S_1,\dots,S_N)}.
\end{equation}
Si quisieramos evaluar la fórmula de la ecuación \ref{eq:mag_total} tal cual esta planteada,
necesitariamos recorrer todos los estados posibles del sistema (al igual que
para calcular exactamente la función de partición, necesaria para calcular las
probabilidades $p_{S_1,\cdots,S_N}$). El problema es que la cantidad de estados
es $2^{N}$, y solo para $N=100$ este número es ridículamente grande. Ninguna
computadora podría recorrer todos estos estados en un tiempo razonable. Pero,
como sabemos, hacer eso tampoco es necesario, ya que solo una fracción de los
estados posibles contribuirá apreciablemente a la suma en la ecuación
\ref{eq:mag_total}. Entonces, dado que no podemos recorrer todos los estados,
una estrategia posible sería solo considerar algunos términos de la ecuación
\ref{eq:mag_total}, de una forma tal que estos términos sean representativos
de los estados que típicamente recorre el sistema cuando se encuentra a
temperatura $\beta^{-1}$. Estos términos, naturalmente, son aquellos para los
cuales la probabilidad $p_{S_1,\dots,S_N}$ es mayor.
Dicho de otra forma, necesitamos `muestrear' la
distribución de probabilidad $p_{S_1,\dots,S_N}$, es decir generar estados de
forma tal que la fracción de veces que se genera el estado $(S_1,\dots,S_N)$
sea $p_{S_1,\dots,S_N}$. El algoritmo de Metrópolis, que pertenece a la familia
mas general de algoritmos de Montecarlo, es una solución a este problema.

\section{Algoritmo de Metrópolis}

El algoritmo de Metrópolis genera estados de la siguiente manera:

\end{document}


